%%
%% This is file `xampl-thesis.tex',
%% generated with the docstrip utility.
%%
%% The original source files were:
%%
%% vakthesis.dtx  (with options: `xampl-thesis')
%%
%% IMPORTANT NOTICE:
%%
%% For the copyright see the source file.
%%
%% Any modified versions of this file must be renamed
%% with new filenames distinct from xampl-thesis.tex.
%%
%% For distribution of the original source see the terms
%% for copying and modification in the file vakthesis.dtx.
%%
%% This generated file may be distributed as long as the
%% original source files, as listed above, are part of the
%% same distribution. (The sources need not necessarily be
%% in the same archive or directory.)
%% xampl-thesis.tex  Приклад головного файла дисертації

\documentclass{vakthesis}


%\documentclass[1space]{vakthesis}  
% Існують кілька опцій, які необхідно вказувати як факультативний
% аргумент команди \documentclass. Наприклад, для докторської
% дисертації необхідно написати
% \documentclass[d]{vakthesis}
% Для бакалаврської \documentclass[b]{vakthesis}, а магістерської \documentclass[m]{vakthesis}


% Підключення необхідних пакетів. Наприклад,
% Пакети AMS для підтримки математики, теорем, спеціальних шрифтів
\usepackage[intlimits]{amsmath}
\allowdisplaybreaks
\usepackage{amsmath,amsthm,amsfonts,amssymb,mathrsfs}

%%%%%%%%%%%%%% маленькі матриці з круглими дужками
\newenvironment{psmallmatrix}
	{\left(\begin{smallmatrix}}
	{\end{smallmatrix}\right)}
%%%%%%%%%%%%%%%%%%

\usepackage{cite}
\usepackage{euscript}

% Налагодження кодування шрифта, кодування вхідного файла
% та вибір необхідних мов

%\usepackage[cp1251]{inputenc}%cp1251, utf-8
%\usepackage[T2A]{fontenc}
\usepackage{fontspec}
\setmainfont{FreeSerif} % або Times New Roman, якщо доступний
\usepackage[english,spanish, ukrainian]{babel}

%%Забираємо слова РОЗДІЛ, а додаємо нумерацію
\usepackage{titlesec}
\titleformat{\chapter}[hang]
  {\normalfont\Large\center\bfseries}
  {\thechapter.}{0.5em}{} % це і є формат: 1. Назва
% Забираємо Розділ зі змісту
% ????????????//



\usepackage{color}
%\usepackage{multind}
\usepackage{multicol}  %my

%моє
\usepackage{index}

  \input{xy}
 \xyoption{all}
 \xyoption{arc}

%\date{\today}


% Налагодження параметрів сторінки (зокрема колонтитиулів).
% Наприклад, за допомогою пакета geometry
\usepackage{geometry}
\geometry{hmargin={30mm,15mm},lines=29,vcentering}


\newtheorem{theorem}{Теорема}[section]
\newtheorem{proposition}[theorem]{Твердження}%[section]
\newtheorem{corollary}[theorem]{Наслідок}%[section]
\newtheorem{lemma}[theorem]{Лема}
\newtheorem{exercise}[theorem]{Задача}

\theoremstyle{definition}
\newtheorem{example}[theorem]{Приклад}%[section]
\newtheorem{construction}[theorem]{Конструкція}%[section]
\newtheorem{remark}[theorem]{Зауваження}%[section]
\newtheorem{definition}[theorem]{Означення}%[section]
\newtheorem{question}[theorem]{Question}


%\setcounter{page}{0} %початок нумерації сторінок з 0, першу компілює порожньою


\renewcommand{\bibname}{СПИСОК ВИКОРИСТАНИХ ДЖЕРЕЛ}
\usepackage{hyperref}


\usepackage{graphicx}
\usepackage{float}

 

\begin{document}

\pagenumbering{gobble}

%\include{title}%           Титулка

% Зміст
\tableofcontents

%\renewcommand{\labelitemi}{$\bullet$} %великі крапочки в списку
\renewcommand{\labelitemi}{{\tiny$\bullet$}} %малі крапочки в списку

\include{general45}  %індекс
% Розділи в окремих файлах

\clearpage
\pagenumbering{arabic}
\setcounter{page}{6}

 \markright{\underline{\it Вступ}}

% \renewcommand{\baselinestretch}{1.3}
 \normalsize
 \chapter*{Вступ}
 \addcontentsline{toc}{chapter}{Вступ}
 %\large
 Глибоке підкріплювальне навчання (Deep Reinforcement Learning, DRL) є 
 однією з найбільш перспективних галузей сучасного штучного інтелекту, 
 яка знаходить широке застосування у різних сферах, включаючи відеоігри. 
 Відеоігри є складними середовищами, де алгоритми DRL можуть демонструвати
  свою здатність до навчання складних стратегій та адаптації до змінних
   умов. Незважаючи на значні досягнення у розвитку DRL, проблема 
   ефективного навчання агентів у складних ігрових середовищах залишається
    актуальною.
 Актуальність даної роботи полягає в дослідженні можливостей різних
  моделей DRL для навчання агентів у грі Space Invaders, яка є класичним
   прикладом складної гри з багатьма викликами для агентів. 
   Розуміння того, яка з моделей (DQN, PPO, A2C) є найбільш 
   ефективною для цієї гри, допоможе вдосконалити методи навчання 
   агентів у подібних середовищах та сприятиме подальшому розвитку технологій DRL.
 
 Метою роботи є порівняння трьох популярних моделей DRL — DQN, PPO та 
 A2C — для визначення їхньої ефективності у навчанні агентів для гри 
 Space Invaders. Це порівняння включає аналіз їхньої продуктивності, 
 стабільності навчання та здатності до адаптації, що дозволить зробити
  вис\-новки про їхню придатність для вирішення складних ігрових завдань.
 
%            Вступ
% *********************** Це є Розділ 1 ************************************


% \setcounter{chapter}{0}
 \chapter{МОДЕЛЮВАННЯ ФІЗИЧНО ПРАВИЛЬНОЇ ВЗАЄМОДІЇ СВІТЛА ІЗ ПРЕДМЕТАМИ}

 
 \par У цьому розділі наводяться необхідні означення та термінологія, а також висвітлено основні проблеми, пов'язані з моделюванням фізично правильної взаємодії світла з 
      об'єктами в комп'ютерній графіці. 

 \section{Людське сприйняття зображень}
  \setcounter{equation}{0}
 \setcounter{theorem}{0}

 \par Перед тим як вдаватись до деталей моделювання фізично правильної взаємодії світла з об'єктами, важливо розглянути, 
 яким чином людське оке сприй\-має світло та як наш мозок формує остаточне зображення.
 \par Світло -- це квантова елетромагнітна хвиля, швидкість якої становить \break $299\,792\,458\,\text{m/s}$. Видимий спектр світла знаходиться між 
 380 та 780 наномет\-рами (рис. \ref{fig:LightSpectrum}). Видиме світло, яке є монохромним, відповідає деякему кольору спектру. Зазвичай джерела світла випромінюють 
 світло у широкому діапазоні довжин хвиль. Наприклад денне світло є суперпозицією деякого діапазону хвиль (рис. \ref{fig:LightSuper}).
У природі спостерігати це явище можна під час виникнення веселки, коли світло проходить через краплі води, розкладаючись при цьому на спектр кольорів.

 \begin{figure}[h]
  \centering
  \includegraphics[scale=1]{Pictures/LightSuper.png}
  \caption{Спектральний розподіл енергії денного світла}
  \label{fig:LightSuper}
\end{figure}

\begin{figure}[h]
\centering
\includegraphics[scale=1]{Pictures/LightSpec.png}
\caption{Видимий спект світла}
\label{fig:LightSpectrum}
\end{figure}

  \par Світло також взаємодіє з навколишніми об'єктами. Припустимо, що світло випромінюється деяким джерелом в однорідних речовинах, наприклад по\-віт\-рі. Оскільки
  рух потоку фотонів є прямолінійним, то деякі промені можуть потрапити одразу в людське око, інші ж -- на об'єкти, які в свою чергу поглинають, відбивають або
  пропускають фотони світла. Відповідно до цього, ми можемо спостерігати різні кольори об'єктів, які залежать від того, які довжини хвиль світла вони відбивають.
  Те світло, яке потрапляє в око людини, проходить через рогівку, кришталик та склоподібне тіло, де воно фокусуються на сітківці ока. Сітківка містить фоторецептори, 
  які реагують на світло і перетворюють його в електричні сигнали, що надсилаються до мозку. Мозок обробляє ці сигнали і формує зображення, яке ми сприймаємо у кінцевому результаті (рис. \ref{fig:LightPath}).

 \begin{figure}[h]
  \centering
  \includegraphics[scale=1]{Pictures/LightPath.png}
  \caption{Шлях світла від джерела до ока людини}
  \label{fig:LightPath}
\end{figure}

\par
Загалом у людини існують дві основні системи фоточутливих рецепторів, що забезпечують зорове сприйняття: палички та колбочки. Розгленемо їх детальніше.
\begin{enumerate}
\item \textbf{Палички} -- фоторецептори, які мають високу чутливість до інтенсивності світла та забезпечують зір при слабкому освітленні 
(скотопічний зір). Вони не розрізняють кольори.
\item \textbf{Колбочки} -- рецептори, відповідальні за кольоровий (фотопічний) зір. Вони чутливі до різних діапазонів довжин хвиль електромагнітного 
випромінювання. Існує три типи колбочок (рис. \ref{fig:LightRecept}):
\begin{enumerate}
    \item \textbf{L-колбочки} (Long) -- реагують на довгі довжини хвиль (при\-близ\-но 560–580 нм), що відповідає червоному діапазону спектра.
    \item \textbf{M-колбочки} (Medium) -- сприймають середні довжини хвиль (при\-близ\-но 530 нм), асоційовані із зеленим кольором.
    \item \textbf{S-колбочки} (Short) -- чутливі до коротких довжин хвиль (при\-близ\-но 420–440 нм), які відповідають синьому кольору.
\end{enumerate}
\end{enumerate}

 \begin{figure}[h]
  \centering
  \includegraphics[scale=1]{Pictures/LightRecept.png}
  \caption{Людські колбочки та діапазон їх чутливості}
  \label{fig:LightRecept}
\end{figure}

Через те, що людське око має лише три типи колбочок, воно не може сприй\-ма\-ти реальний спектральний розподіл енергії світла,
завдяки цьому око мож\-на обманювати, і різні спектрані розподіли енергії світла можуть візуалізуватись як однакові кольори. Саме через це, в комп'ютерній графіці
застосовується RGB кольорова модель, де R -- red (червоний), G --green (зелений), B -- blue (синій). Змішуючи ці три кольори в різних пропорціях, можна отримати 
більшість кольорів, які сприймаються людським оком (рис. \ref{fig:RGB}). Прямий фі\-зич\-ний зв'язок між спектральним розподілом енергії світла та RGB описано в \cite{Ch1} та \cite{Ch2}. У контексті комп’ютерної графіки, зокрема при \texit{рендерингу} -- процесі створення фотореалістичного або нефотореалістичного зображення на основі вхідних даних -- це має вирішальне значення.
Більшість сучасних моніторів та екранів використовують sRGB кольорову модель для відображення зображень, але для фізично правильного рендерингу треба працювати в 
лінійному спектрі, для цього застосовується так звана \textit{гамма ко\-рек\-ція}, яка дозволяє перетворити кольори з sRGB в лінійний спектр \cite{Ch5}.
Основна ідея, яку слід засвоїти, полягає в тому, що немає потреби симулювати транспортування світ\-ла для кожної довжини хвилі окремо. Для більшості прак\-тич\-них застосувань 
достатньо розраховувати перенесення світ\-ла для трьох основних кольорів. Це значно спрощує обчислення та робить задачі 
рендерингу обчислювально ефек\-тив\-ні\-ши\-ми.

\begin{figure}[h]
  \centering
  \includegraphics[scale=1]{Pictures/RGB.png}
  \caption{RGB кольорова модель}
  \label{fig:RGB}
\end{figure}

 \section{Визначення рівняння рендерингу}
   \setcounter{equation}{0}
 \setcounter{theorem}{0}

 \subsection{Радіометричні величини} \\

\par
Перш ніж дати визначення рівнянню рендерингу, варто розглянути деякі основні фізичні величини, які в ньому зу\-стрі\-ча\-ють\-ся. Одним із базових понять є \textit{тілесний кут}.

\paragraph{Тілесний кут.}
\par
У шкільній програмі вводиться поняття кута у двовимірному просторі. Щоб визначити, який кут охоплює об’єкт з певної точки спостереження, уявімо коло з центром у цій точці та проектуємо об’єкт на нього. Кут визначається як відношення довжини дуги $s$ до радіуса $r$:
\[
\theta = \frac{s}{r}.
\]
Одиниця плоского кута називається “радіан” (рад) і дорівнює кутові, для якого $s/r=1$.
\par
У тривимірному просторі аналогом звичайного кута є \textit{тілесний кут} (англ. \textit{solid angle}) $\omega$ --  це частина простору, обмежена певною незамкненою конічною поверхнею. 
\par
Щоб визначити тілесний кут, який охоплює об’єкт з певної точки, роз\-міс\-ти\-мо у цій точці центр уявної сфери та спроектуємо об’єкт на її поверхню. Таким чином, тілесний кут визначається як відношення площі проєкції $s$ до квадрата радіуса сфери:
\[
\omega = \frac{s}{r^2}.
\]
Одиницею тілесного кута є стерадіан \textit({sr}),  для якого $\frac{S}{r^2}=1$. Повна сфера має тілесний кут $4\pi$ стерадіан.

\par
Для обчислення тілесного кута, що покриває певну область, необхідно спроектовану область розбити на нескінченно малі елементи $d\omega$ й інтегрувати по всій області:
\[
\omega = \int d\omega.
\]

\par
Зручним способом параметризації є сферичні координати, які задаються двома кутами: полярним $\theta$ (від $0$ до $\pi$) та азимутальним $\phi$ (від $0$ до $2\pi$). Нескінченно мала ділянка тілесного кута виражається через поверхневий елемент $ds$:
\[
d\omega = \frac{ds}{r^2}.
\]
Припускаючи, що радіус $r = 1$, обчислимо $ds$ як площу елемента поверхні сфери:
\[
ds = \sin{\theta} \, d\theta \, d\phi.
\]
Таким чином, фінальна формула для нескінченно малої ділянки тілесного кута набуває вигляду:
\[
d\omega = \sin{\theta} \, d\theta \, d\phi.
\]

А загальна формула для тілесного кута, що охоплює певну область на сфері, виглядає так:
\begin{equation}
\label{eq:SolidAngle}
\omega = \int_0^{2\pi} \int_0^{\pi} \sin{\theta} \, d\theta \, d\phi.
\end{equation}

\paragraph{Потік випромінювання (Radiant Flux).}
\textit{Потік випромінювання} \linebreak (англ. \textit{radiant flux})~$\Phi$, -- це енергія електромагнітного випромінювання, що пе\-ре\-дає\-ть\-ся в одиницю часу:
\[
\Phi = \frac{dQ}{dt},
\]
де $Q$ -- енергія, $t$ -- час. Одиниця вимірювання потоку енергії -- ват (Вт), тобто джоуль на секунду (Дж/с).

Кожен фотон несе енергію, яку можна обчислити за формулою:
\[
E = \frac{hc}{\lambda},
\]
де $h$ -- стала Планка, $c$ -- швидкість світла, $\lambda$ -- довжина хвилі.

Таким чином, потік випромінювання можна інтерпретувати як сумарну енергію фотонів, що випромінюються джерелом світла за одиницю часу. У графіці $\Phi$ 
використовується для опису загального випромінювання точкового джерела.

\paragraph{Інтенсивність випромінювання (Radiant Intensity).}
\textit{Інтенсивність} \linebreak \textit{вип\-ро\-мі\-ню\-ван\-ня} (англ. \textit{Radiant Intensity})  $I$ -- це потік випромінювання на одиницю тілесного кута:
\[
I = \frac{d\Phi}{d\omega}.
\]
Одиниця вимірювання інтенсивності випромінювання -- Вт/ср (ват на стерадіан). Ця величина використовується, коли джерело випромінює світло нерівномірно в різних напрямках, як, наприклад, 
у випадку прожектора. Для точкового джерела, що випромінює рівномірно у всіх напрямках:
\[
I = \frac{\Phi}{4\pi}.
\]

\paragraph{Опроміненість (Irradiance).}
\textit{Опроміненість} (англ. \textit{Irradiance}) $E$ -- це потік випромінювання, що потрапляє на одиницю площі поверхні:
\[
E = \frac{d\Phi}{dA}.
\]
Одиниця вимірювання -- Вт/м$^2$. Потік може надходити з усіх напрямків півсфери над поверхнею. Для нескінченно малої ділянки $dA$, яку 
освітлює точ\-ко\-ве джерело, враховується тілесний кут $d\omega$, під яким джерело можна побачити з цієї ділянки. Якщо $\theta$ -- кут між нормаллю до 
поверхні та напрямком на джерело, то проєкція враховується через множник $\cos{\theta}$:

\[
E = \frac{I \cdot \cos{\theta}}{r^2},
\] де $r$ -- відстань до джерела.

Оскільки $I = \frac{\Phi}{4\pi}$, то для точкового джерела маємо:
\[
E = \frac{\Phi \cdot \cos{\theta}}{4\pi r^2}.
\]

Це показує, що опроміненість зменшується обернено пропорційно до квад\-ра\-та відстані до джерела випромінювання та залежить від кута падіння світла.

\paragraph{Енергетична яскравість (Radiance).}
\texit{Енергетична яскравість} \linebreak (англ. \textit{Radicance}) $L$ визначається як потік випромінювання, що проходить через одиницю площі в певному напрямку, на одиницю тілесного кута:
\[
L = \frac{d^2\Phi}{dA_{\perp} \, d\omega},
\]
де $dA_{\perp} = \cos{\theta} \, dA$ -- проєкція площі в напрямку потоку. Одиниця вимірювання -- $\text{Вт}/(\text{м}^2\cdot\text{ср})$.
Ця величина не залежить від відстані до джерела, адже при зменшенні відстані тілесний кут збільшується, але площа яка спостерігається -- зменшується.
\par
Спостергіти це явище ми можемо на прикладі стіни, змінючи дистанцію до неї, її яскравість 
є сталою.

\paragraph{Зв'язок між опроміненістю та енергетичною яскравістю.}
Опроміненість можна визначити через інтегрування енергетичної яскравості по всій півсфері $\Omega_h$ напрямків над поверхнею:
\begin{equation}
 \label{eq:RadianceToIrradiance}
    E = \int_{\Omega_h} L(\omega) \cos{\theta} \, d\omega.
\end{equation}

\subsection{Рівняння рендерингу} \phantom{Латех просто не скидати текст на новий рядок}
\par Рівняння рендерингу описує, як світло взаємодіє з поверхнями об'єктів і як це впливає на зображення, яке ми бачимо. Воно базується на фізичних принципах 
передачі світла та його взаємодії з матеріалами.
\begin{equation}
\label{eq:RenderingEquation}
  L_o(\vec{v}) = L_e(\vec{v}) + \int_{\Omega_h} f_r(\vec{v},\vec{l} ) L_i(\vec{l}) \cos{\theta} d\omega,
\end{equation} де: $L_o(\vec{v})$ -- вихідна енергетична яскравість у напрямку $\vec{v}$, $L_e(\vec{v})$ -- енер\-ге\-тич\-на яскравість, що випромінюється поверхнею в напрямку $\vec{v}$,
$f_r(\vec{v},\vec{l})$ -- \textit{двопроменева функція розподілу відбивної здатності} (BRDF), що описує, як світло з напрямком $\vec{l}$ відбивається від 
поверхні в напрямок $\vec{v}$, $L_i(\vec{l})$ -- вхідна енер\-ге\-тич\-на яскравість у напрямку $\vec{l}$, 
$\theta$ -- кут між нормаллю до поверхні та напрямком на джерело світла, $d\omega$ -- тілесний кут.
\par
Якщо придивитися уважно, то можна побачити, що рівняння рендерингу \ref{eq:RenderingEquation} містить в собі рівняння для опроміненості \eqref{eq:RadianceToIrradiance}. Інакше кажучи, рівняння \ref{eq:RenderingEquation} описує, що саме побачить спостерігач у напрямку $\mathbf{-v}$, дивлячись на точку, яка сама 
випромінює світло у напрямку $\vec{v}$ та додатково опромінюється іншими об'єктатами, розташованих на уявній півсфері, що охоплює цю точку.

\paragraph{Складність обчислення рівняння освітлення.}

Інтегрування по всій півсфері напрямків~$\Omega_h$ означає врахування всіх можливих напрямків по\-ши\-рен\-ня світла, яке може впливати на задану точку поверхні.
Це робить рівняння освітлення обчислювально складним, адже для кожної точки сцени потрібно проінтегрувати внесок світла з усіх напрямків півсфери, орієнтованої відносно нормалі до поверхні.
Під \textit{сценою} у комп’ютерній графіці розуміють сукупність усіх візуальних елементів -- геометрії, матеріалів, джерел світла та камери -- які разом утворюють просторову конфігурацію, що підлягає візуалізації.

Кожен об’єкт у сцені сам по собі є джерелом випромінювання, тобто володіє певною вихідною енергетичною яскравістю в довільному напрямку~$\vec{v}$. Взаємне 
освітлення між об’єктами створює глобальну взаємозалежність, де світ\-ло, відбите від однієї поверхні, може впливати на інші поверхні, і так далі -- потенційно нескінченну
 кількість разів.

Складність посилюється тим, що будь-яка поверхня є зв'язною\footnote{Кількість світлових променів у сцені є скінченними, проте їхня кількість настільки велика, що доречно вважати їх нескінченними} множиною точок, а з кожної точки в нескінченну кількість напрямків можуть надходити фотони. На 
локальому рівні поверхні не ідеально гладкі: вони можуть мати шорсткості, мікрогеометрію, відмінні оптичні властивості (наприклад, металічність, діелектричність,
 шорсткість тощо), що впливає на спосіб взаємодії зі світлом.

Оскільки світло може багаторазово відбиватися між поверхнями перед тим, як потрапити в камеру або око спостерігача, то точне симулювання всіх траєкторій кожного 
фотона є обчислювально недосяжним\footnote{Слід зауважити, що існують методи, такі як Ray Tracing та Path Tracing, які симулюють поведінку світла, але лише для скінченної кількості променів, та
малої кількість відбитів світла від поверхонь} завданням для су\-час\-них комп’ютерів.

У зв’язку з цим, замість точного моделювання всіх траєкторій, у комп’ю\-тер\-ній графіці застосовуються стохастичні методи, які описують розповсюдження світла у вигляді
 ймовірнісного процесу. Зокрема, розглядається ймовірність того, що промінь світла з певного напрямку~$\vec{l}$, взаємодіючи з поверхнею з відомими матеріальними 
 властивостями, буде відбитий в інший напрямок або поглинутий. Саме BRDF функція дозволяє моделювати цю ймовірність, і давати загальне уявлення про те, як світло
взаємодіє з поверхнею.

\subsection{BRDF та його роль у моделюванні відбиття світла}
\phantom{Латех...}
\par BRDF (Bidirectional Reflectance Distribution Function -- двонаправлена функ\-ція розподілу відбиття) -- це чотиривимірна функція, якщо не враховувати залежність 
від довжини хвилі. Саме вона точно описує відбивні властивості поверхні. BRDF певного матеріалу можна виміряти експериментально й зберігати у вигляді 4D-таблиці. 
Однак такий підхід потребує великої кількості пам'яті та ускладнює редагування матеріалів. Тому у більшості випадків використовують параметричні моделі BRDF, як-от 
модель Фонга чи модель Бліна-Фонга \cite{Ch3}.
\par У комп'ютерній графіці також застосовується поняття мікрофасетів -- це дрібні ділянки поверхні, розмір яких набагато менший за піксель. 
Передбачається, що кожен мікрофасет поводиться як ідеальне дзеркало, а їх орієнтації розподілені за певним статистичним законом розподілу навколо мікро\-ско\-піч\-ної нормалі поверхні. Через ці відхилення від ідеальної нормалі світло відбивається не в одному напрямку, а розсіюється навколо напрямку ідеального відбиття.
Чим вища шорсткість поверхні, тим більший розмах розподілу відбитого світла. Для металів фотони або відбиваються, або поглинаються всередині матеріалу -- залежно від 
довжини хвилі. Через це відбите світло може мати певний колір, як у випадку з  міддю чи золотом. Така поведінка характерна саме для металів.У діелектриків 
(неметалевих матеріалів) модель відбиття інша, ніж у металів. Частина фотонів також відбивається від поверхні -- це так зване дзеркальне відбиття (specular reflection). 
Але на відміну від металів, це відбиття не залежить від довжини хвилі, тому воно не має кольору. Решта фотонів проникає в матеріал, де або поглинається, 
або розсіюється під поверхнею в довільних напрямках. Частина з них зрештою виходить назад з поверхні -- це явище називається дифузним відбиттям (diffuse reflection). Воно залежить від довжини хвилі, а отже, для діелектриків є кольоровим.
\par
Єдиного вигляду у функції BRDF нема, адже це сімейство функцій які задаються формулою:
\begin{equation}
\label{eq:BRDF_DS}
f_r = f_d + f_s,
\end{equation}
де $f_d$ -- це дифузне відбиття, а $f_s$ -- дзеркальне відбиття, та повинні задовольняти низку фізичних властивостей:

\begin{enumerate}
    \item \textbf{Невід’ємність:} BRDF ніколи не повинна генерувати від'ємне зна\-чен\-ня, адже від'ємної енергетичної яскравості не існує.

    \item \textbf{Реципрокність (взаємність) за Гельмгольцем:} Зна\-чен\-ня BRDF повинне залишатися незмінним при перестановці 
    напрямків спостереження $\vec{v}$ і освітлення $\vec{l}$, тобто:
    \[
    f_r(\vec{v}, \vec{l}) = f_r(\vec{l}, \vec{v}).
    \]
    Це означає, що освітлення сцени і її спостереження повинні бути симетричними по відношенню до розподілу світлового потоку.

    \item \textbf{Закон збереження енергії:} BRDF повинна також дотримуватися принципу збереження енергії. Якщо розглянути другу частину рів\-нян\-ня 
    \ref{eq:RenderingEquation}, яка описує інтегрування опроміненості по напівсфері $\Omega$, та припустити, що вхідна опроміненість з усіх напрямків дорівнює 1, 
    то енергетична яскравість не може перевищувати цього зна\-чен\-ня:
    \[
    \int_{\Omega} (f_d(\vec{v}, \vec{l}) + f_s(\vec{v}, \vec{l})) \cdot\cos\theta \, d\omega \leq 1.
    \]
    Значення, менше за 1, є фізично допустимим, оскільки частина світ\-ла може бути поглинена матеріалом і перетворена, наприклад, у теп\-ло. 
    Проте значення, що перевищують 1, порушують закон збереження енергії та вказують на некоректність BRDF, адже у такому випадку модель фактично створює нову 
    енергію, що є неможливим при прос\-то\-му відбитті світла.
\end{enumerate}%         Розділ 1 
% *********************** Це є Розділ 2 ************************************

 %\markright{\underline {\it Розділ 2. Теоретичне рішення проблеми}}

 %\setcounter{chapter}{1}
 \chapter{ТЕОРЕТИЧНЕ ВИРІШЕННЯ ПРОБЛЕМИ}

 
 У цьому розділі розглядається теоретичне підґрунтя розв'язання задачі рендерингу в рамках створення власного графічного рушія. Спочатку описується 
 базовий процес рендерингу геометрії, далі розглядаються моделі BRDF, що визначають фізично коректне відбиття світла від поверхонь, а також методи оптимізації, 
 які забезпечують ефективну роботу системи в реальному часі.

\section{Рендеринг геометрії}
  \setcounter{equation}{0}
 \setcounter{theorem}{0}
 Рендеринг геометрії є фундаментальною складовою комп’ютерної графіки й одночасно однією з найскладніших задач. Безпосереднє математичне описання складних форм (наприклад, анатомії людини) аналітичними функціями практично неможливе, тому використовуються чисельні апроксимації. Найпоширеніший підхід полягає в розбитті поверхні на набір елементарних багатокутників, зазвичай -- трикутників. Така триангуляція дозволяє будь–який многовид наближати довільно точно, регулюючи щільність сітки.

Графічний процесор (GPU) оптимізований під ефективну обробку саме трикутних сіток. Кожен трикутник задається трьома вершинами та відповідними індексами ребер. GPU приймає буфери вершин і індексів, виконує трансформації та проєкцію, а потім передає дані далі в конвеєр -- тобто в pipeline.

\subsection*{Графічний конвеєр і шейдери}
Графічний пайплайн складається з кількох етапів:
\begin{enumerate}
    \item \textbf{Вершинний шейдер (Vertex Shader)} -- застосовує афінні пе\-рет\-во\-рен\-ня до вершин: поворот, масштаб, перенесення та проєкцію у відсічений об’єм.
    \item \textbf{Тесселяція (опціонально)} -- процес динамічного розбиття примітивів (переважно трикутників або чотирикутників), що формують геометрію об'єктів, на менші  з метою підвищення рівня деталізації. 
      
    \item \textbf{Геометричний шейдер (Geometry Shader)} -- може додатково генерувати чи модифікувати примітиви на основі вхідних даних.
    \item \textbf{Растеризація} -- перетворює примітиви (трикутники) у фрагменти (пік\-се\-лі), відсікає ті, що знаходяться поза областю перегляду.
    \item \textbf{Фрагментний (піксельний) шейдер (Fragment / Pixel Shader)} -- обчислює колір кожного фрагмента з урахуванням матеріальних влас\-ти-\-вос\-тей і текстур.
    \item \textbf{Тест глибинності й блендінг} -- вирішує, які фрагменти за\-ли\-шаю\-ть\-ся у фінальному буфері кадру.
\end{enumerate}

\subsection*{Перспективна проєкція та відсікання.}
Для реалізації правдоподібного відтворення сцени необхідно врахувати сприйняття глибинних відстаней людиною. \textit{Перспективна проєкція} формально задається усіченою пірамідою, яку можна визначити матрицею:
\[
P = 
\begin{pmatrix}
\frac{1}{\tan(\tfrac{fov}{2})\,a} & 0 & 0 & 0 \\
0 & \frac{1}{\tan(\tfrac{fov}{2})} & 0 & 0 \\
0 & 0 & \frac{z_\mathrm{far}+z_\mathrm{near}}{z_\mathrm{near}-z_\mathrm{far}} & \frac{2\,z_\mathrm{far}\,z_\mathrm{near}}{z_\mathrm{near}-z_\mathrm{far}} \\
0 & 0 & -1 & 0
\end{pmatrix},
\]
де $fov$ -- кут огляду, $a$ -- аспектне співвідношення екрану, $z_\mathrm{near}$ і $z_\mathrm{far}$ -- межі відсічення. Перспектива забезпечує зменшення розмірів 
віддалених об’єктів і дає змогу відсікати геометрію поза ближня і дальня площини відсікання, що знижує навантаження на конвеєр.

\subsection*{Текстурування і матеріали.}
Окрім геометрії, для реалістичного виг\-ля\-ду необхідно вказати:
\begin{itemize}
    \item \textbf{Колір (Albedo)} -- базова текстура дифузного відбиття.
    \item \textbf{Шорсткість(Roughness) / Металічність(Metalness)} -- парамет\-ри для PBR-моделювання, які відповідають за фізичні властивості матеріалу.
    \item \textbf{Текстурування нормалей(Normal Map)} -- для імітації дрібних нерівностей, за рахунок модифікування нормалі точок поверхні без збільшення полігонів.
\end{itemize}

У піксельному шейдері ці карти комбінуються з параметрами світла та BRDF, щоб отримати остаточний колір пікселя.

\medskip
Таким чином, рендеринг геометрії охоплює низку послідовних етапів: від апроксимації форми трикутниками, через трансформації та відсікання, до текстурування й 
шейдерної обробки, причому на кожному кроці необхідно балансувати між якістю зображення та продуктивністю системи.

\section{Мікрофасетна модель відбиття}
\setcounter{equation}{0}
\setcounter{theorem}{0}

У цьому підрозділі розглядається мікрофасетна модель відбиття світла \linebreak Cook-Tor\-ran\-ce BRDF. Вперше ця модель була запропонована 
в 1982 році у праці Роберта Кука та Кеннета Торренса \cite{cook1982reflectance} як фізично обґрунтована альтернатива емпіричним моделям відбиття, таким як Фонг або Блінн–Фонг.

\subsection{BRDF}\\
\par
Основна ідея BRDF полягає у тому, що поверхня моделюється як сукупність уявних мікрофасетів. Кожна така мікрофасета роз\-гля\-да\-єть\-ся як ідеальне дзеркало. Поведінка макроскопічної поверхні описується статистично 
через розподіл орієнтацій цих мікрофасетів.

\par
Цей підхід дозволяє значно точніше апроксимувати фізичні властивості реальних матеріалів порівняно з класичними моделями, що були раніше поширені в
 рендерінгу у реальному часі. Наприклад, модель Фонга була популярною завдяки своїй обчислювальній простоті. Проте із розвитком обчислювальної техніки 
 галузь комп'ютерної графіки майже повністю перейшла на фізично обґрунтований рендеринг (Physically Based Rendering, PBR), де мікрофасетні моделі відіграють
  ключову роль як в офлайн-рендерінгу, так і в рендерінгу в реальному часі.

\par
На конференції SIGGRAPH 2012 року \cite{Ch6} Брент Берлі, дослідник з \textit{Walt Disney Animation Studios}, запропонував спрощений інтерфейс для
 використання мікрофасетної моделі BRDF. У цьому підході всі параметри нормалізовано до інтервалу $[0;1]$, що суттєво спрощує взаємодію з моделлю. 
Цей підхід відомий як \textit{metallic–roughness workflow} (металево-шорсткий підхід) і нині є де-факто стандартом для опису матеріалів у комп’ютерній графіці.

Модель ґрунтується на двох основних параметрах:
\begin{itemize}
    \item \textbf{Металевість} (\textit{metallic}) -- визначає тип матеріалу: метал (\textit{metallic} = 1) або діелектрик (\textit{metallic} = 0). Метали, 
    такі як золото, срібло чи мідь, мають високу дзеркальну відбивну здатність. Діелектрики (наприклад, пластик, дерево, гума) мають інший характер відбиття світла.
    
    \item \textbf{Шорсткість} (\textit{roughness}) -- описує розсіювання мікрофасетів. Для ідеально гладких поверхонь (наприклад, полірування) \textit{roughness} = 0,
     і всі мікрофасети орієнтовані однаково, що створює дзеркальне відбиття. Із зростанням шорсткості орієнтації мікрофасетів стають випадковішими, що спричиняє 
     більш розсіяне відбиття.
\end{itemize}

\begin{figure}[h]
  \centering
  \includegraphics[scale=0.4]{Pictures/rough-metal.png}
  \caption{Металево-шорсткий підхід}
  \label{fig:RM}
\end{figure}

\par
Хоча модель Disney дозволяє використовувати дробові значення па\-ра\-мет\-ра \textit{metallic} (наприклад, $0.5$), фізично це не має сенсу, оскільки матеріал не може 
одночасно бути і металом, і діелектриком. Проте у практичних випадках, особливо при використанні текстур з обмеженою роздільною здатністю, у межах одного 
пікселя може міститися інформація про кілька матеріалів. У таких ситуаціях дробові значення \textit{metallic} інтерпретуються як середнє значення між різними типами 
матеріалів.

\par
Ще одним важливим параметром у моделі є \textbf{базовий колір} (\textit{base color}) рис. (\ref{fig:BC}). Це вектор з трьох компонентів (RGB), 
який відіграє різну роль за\-леж\-но від типу матеріалу.

\begin{itemize}
    \item Для \textbf{діелектриків} (неметалів) цей параметр визначає альбедо ма\-те\-ріа\-лу -- тобто частку світла, яка дифузно відбивається від поверхні. Значення кожної 
    компоненти RGB знаходиться в межах інтервалу $[0;1]$.
    
    \item Для \textbf{металів}, навпаки, \texit{base color} представляє значення коефіцієнта відбиття Френеля при нормальному падінні (Fresnel reflectance), яке 
    характеризує дзеркальну компоненту відбитого світла. Докладний розгляд закону Френеля буде подано у наступному підрозділі.
\end{itemize}

\begin{figure}[h]
  \centering
  \includegraphics[scale=0.6]{Pictures/BaseColor.png}
  \caption{Базовий колір матеріалів}
  \label{fig:BC}
\end{figure}

Крім того, у моделі передбачено ще один параметр -- \textbf{відбивна здатність} (\textit{reflectance}) рис. (\ref{fig:Specular}), який застосовується лише для діелектричних матеріалів. 
Він також пов’язаний із законом Френеля, однак для неметалів. Цей параметр визначає інтенсивність дзеркальної компоненти відбиття (specular ref\-lec\-tion), 
притаманної діелектрикам.

На відміну від металів, де дзеркальне відбиття при нормальному падінні світла може досягати майже $100\%$, для діелектриків цей показник значно нижчий -- він 
зазвичай перебуває в межах від 0\% до 16\% залежно від матеріалу та кута падіння, та за для зручності ці межі відображають у відрізок $[0,1]$.

\begin{figure}[h]
  \centering
  \includegraphics[scale=0.75]{Pictures/Specular.png}
  \caption{Відбивна здатність}
  \label{fig:Specular}
\end{figure}

\par
У рівнянні \ref{eq:RenderingEquation} BRDF виступає у ролі функці $f_r$. Так як BRDF належить до сімейства функцій,
точного вигляду вона немає. В даній квалфікаційній роботі розглядається Cook-Torrance BRDF модель, яка має такий вигляд:
\begin{equation}
\label{eq:BRDF}
f_r(\vec{v},\vec{l}) = \frac{\rho_d}{\pi} + \frac{\mathbf{F}(\vec{v},\vec{h})\cdot\mathbf{D}(\vec{h})\cdot\mathbf{G}(\vec{l},\vec{v})}{4 \cdot\langle \vec{n}, \vec{l}\rangle \cdot\langle \vec{n}, \vec{v}\rangle},
\end{equation}
де, $\vec{v}$ -- це вектор спостереження, тобто вектор у напрямку ока, $\vec{l}$ напрямок промення світла, $\vec{h}$ напівнапрямленний вектор, $\vec{n}$ нормаль поверхні та $\rho_d$ деяка константа.
Тут перший доданок виступає у ролі $f_d$, а другий у ролі $f_s$ рівності \ref{eq:BRDF_DS}. Окрім того, $\mathbf{F}(\vec{v},\vec{h})$ -- відбиття Френеля, $\mathbf{D}(\vec{h})$ -- функція
розподілу нормалей та $\mathbf{G}(\vec{l},\vec{v})$ -- геометричний фактор.

\subsection{Ефект Френеля}
 \setcounter{equation}{0}
 \setcounter{theorem}{0}

Один із ключових етапів побудови фізично коректної моделі відбиття світла полягає у врахуванні \textit{ефекту Френеля}, який описує залежність коефіцієнта 
відбиття від кута падіння світла та показників заломлення матеріалів на межі поділу середовищ.

\par
Ефект Френеля легко можна спостерігати в реальному житті. Наприклад, на рис.\ref{fig:FresnelLake} зображено озеро з абсолютно рівною поверхнею води; у нижній час\-ти\-ні чітко видно дно. 
Натомість у верхній частині видно лише відбиття. Така різниця пояснюється тим, що співвідношення між переданим і відбитим світ\-лом не є сталим: воно змінюється 
залежно від кута падіння променя та оп\-тич\-них властивостей (показників заломлення) матеріалів.

Коли ми дивимося у воду під прямим кутом, лише незначна частина світла відбивається, решта проходить крізь поверхню води, і ми бачимо дно. Проте зі збільшенням кута
 (наближенням до дотичного спостереження), частка відбитого світла зростає, а частка переданого -- зменшується. Тобто при малих кутах падіння спостерігається в основному 
 пропускання, а при великих -- відбиття.

 \begin{figure}[h]
  \centering
  \includegraphics[scale=0.25]{Pictures/fresnelLake.jpg}
  \caption{Ефект Френеля в реальному житті}
  \label{fig:FresnelLake}
\end{figure}

\newpage
На рис. \ref{fig:Snells} зображено типову ситуацію взаємодії світла з поверхнею на межі двох середовищ. У випадку басейну це межа повітря 
(з показником заломлення $\eta_1$) і води (з більшим показником заломлення $\eta_2$). Кут падіння $\theta_1$ визначається відносно нормалі до поверхні. Закон 
відбиття стверджує, що кут між нормаллю та відбитим променем дорівнює $\theta_1$. Для заломленого (переданого) променя кут $\theta_2$ є меншим, оскільки світло 
переходить із менш густого середовища у густіше (тобто промені заломлюються у бік нормалі).

Цей зв’язок описується законом Снеліуса:
\begin{equation*}
    \eta_1 \sin \theta_1 = \eta_2 \sin \theta_2,
\end{equation*}
де $\eta_1$ та $\eta_2$ -- показники заломлення відповідних середовищ, $\theta_1$ -- кут падіння, а $\theta_2$ -- кут заломлення.


 \begin{figure}[h]
  \centering
  \includegraphics[scale=0.75]{Pictures/Snells.png}
  \caption{Заломлення світла}
  \label{fig:Snells}
\end{figure}

Знаючи $\eta_1$, $\eta_2$ та $\theta_1$, ми можемо обчислити $\theta_2$ і, у подальшому, розрахувати частку відбитого світла за допомогою рівнянь Френеля:
\begin{equation*}
F_{para} = \frac{\eta_2\cos\theta_1 - \eta_1\cos\theta_2}{\eta_2\cos\theta_1 + \eta_1\cos\theta_2};
\end{equation*}
\begin{equation*}
F_{pere} = \frac{\eta_2\cos\theta_2 - \eta_1\cos\theta_1}{\eta_2\cos\theta_2 + \eta_1\cos\theta_1};
\end{equation*}
\begin{equation}
\label{eq:fresnel}
F = \frac{1}{2}(F_{para} + F_{pere}).\quad
\end{equation}

Частка переданого світла визначається як:
\begin{equation*}
    T = 1 - R,
\end{equation*}
де $R$ -- коефіцієнт відбиття, який залежить від поляризації світла, кута падіння та оптичних властивостей матеріалів.
\paragraph{Відбиття Френеля для діелектриків та металів.}

\par 
Світлові фотони при взаємодії з поверхнею можуть бути відбитими із певними йомовірностями, які напряму залежать від кута падіння. Для діе\-ле\-кт\-ри\-ків, відбита частина світла розсіюється мікрофасетками поверхні, що мак\-ро\-ско\-піч\-но формує дзеркальну складову відбиття. Передана частина зазнає внут\-ріш\-ніх 
 розсіянь, частково поглинається і випромінюється у випадкових напрямках. Це створює дифузну складову.
Чим більше світла відбивається на поверхні, тим менше його проникає всередину, а отже, дифузна складова стає слабшою. Оскільки ймовірність поглинання світла 
залежить від довжини хвилі, то дифузна частина є кольоровою. Натомість дзеркальна складова зазвичай є некольоровою для діелектриків.


\par
Для металів ситуація інакша: передана частина повністю поглинається, тому дифузна складова відсутня. Натомість дзеркальна складова є кольоровою, оскільки залежить
 від довжини хвилі. Це пояснює, чому метали мають ха\-рак\-тер\-не забарвлення у дзеркальних відбиттях.

 \par
Для перпендикулярного падіння ($\theta_1 = 0^\circ$), рівняння Френеля значно спрощуються. Якщо припустити, що $\eta_1 = 1$ (повітря або вакуум), а $\eta_2 = 1.5$ 
(наприклад, скло), то за формулою \ref{eq:fresnel} можна обчислити значення $F_0 \approx 0.04$, тобто $4\%$ світла буде відбито, а 96\% -- пропущено.

\begin{equation*}
    F_0 = \left( \frac{\eta_2 - \eta_1}{\eta_2 + \eta_1} \right)^2.
\end{equation*}

\par 
У мікрофасетній моделі поверхня складається з великої кількості кри\-хіт\-них ідеально плоских дзеркал. Через це для розрахунку відбиття Френеля ми використовуємо не кут між 
вхідним променем та нормаллю поверхні, а кут між напрямком огляду (або світла) і вектором півшляху (halfway vector). У данному випадку
вектор півшляху використовується як нормаль мікрофасеток через те, що нас цікавить частина світла яка відбивається у напрямок спостерігача:

\begin{equation}
    \cos \theta = \langle\vec{v}, \vec{h}\rangle.
\end{equation}

Тому саме цей кут $\theta$ використовується у рівняннях Френеля в мікрофасетній BRDF-моделі, запропонованій Куком і Торренсом (Cook \& Torrance, 1982) \cite{cook1982reflectance}, та
ефект Френеля обчислюється наступним чином:
\begin{equation}
  F_{Cook-Torrance}(\vec{v},\vec{h}) = \frac{1}{2}(\frac{g - c}{g + c})^2(1 + (\frac{c(g+c)-1}{c(g-c)+1})^2),
\end{equation}
де $c = \langle\vec{v},\vec{h}\rangle$ та $g = \sqrt{\eta_2^2+c^2-1}$.
Незважаючи на відсутність тригонометричних функцій, ці обчислення є дорогими. Щоб зменшити обчислювальні витрати, часто використовують \textit{апроксимацію 
Шліка (aнгл. Schlick's approximation)}:

\begin{equation}
    F_{Schlick}(\theta) = F_0 + (1 - F_0)(1 - \cos \theta)^5.
\end{equation}

Ця формула дозволяє швидко наближати поведінку функції Френеля без втрати надто великої точності.
 Як показано на рис. \ref{fig:Schlick}, при $\theta = 0^\circ$ вона точно дає $F = F_0$, а при $\theta = 90^\circ$ -- $F = 1$. 
 У проміжних кутах апроксимація лише незначно відхиляється від точного розв’язку.
 \begin{figure}[h]
  \centering
  \includegraphics[scale=0.55]{Pictures/Shlick.png}
  \caption{Порівняння апроксимацію Шліка}
  \label{fig:Schlick}
\end{figure}

Цей підхід широко використовується у реальному часі (наприклад, у PBR-шейдерах), забезпечуючи добрий баланс між точністю і швидкодією.

\subsection{Функція розподілу нормалей}\\

\par
Як вже зазначалося раніше, поверхня фізичного матеріалу має мікроскопічну шорсткість, яка складається з безлічі мікрофасетів із власними мікронормалями. 
Ця мікроструктура відіграє ключову роль у формуванні зовнішнього вигляду поверхні під час взаємодії зі світлом.

Розподіл таких мікронормалей описується за допомогою функції розподілу нормалей (NDF, \textit{Normal Distribution Function}). Вона є складовою частиною 
двонапрямної функції розподілу відбиття, яка визначає, як світло від\-би\-ва\-єть\-ся з поверхні y 
заданому напрямку.

BRDF формалізує кількість енергії, що відбивається від поверхні в напрямку спостерігача від одного променя вхідного світла. Вона є зваженою величиною, оскільки 
враховує NDF, яка насправді є функцією розподілу ймовірностей (PDF, \textit{Probability Distribution Function}) для напрямків мікронормалей.

У випадку ідеально гладкої, полірованої поверхні всі мікрофасети орієнтовані в одному напрямку -- відповідно до макроскопічної нормалі поверхні. Натомість 
при більш шорсткій поверхні орієнтації мікрофасетів статистично розподілені навколо цієї нормалі. Відповідно, відбиття світла також розсіюється навколо ідеального
 дзеркального напрямку. Із зростанням шорсткості поверхні ступінь цього розсіювання збільшується рис. (\ref{fig:NDF}).

  \begin{figure}[h]
  \centering
  \includegraphics[scale=1]{Pictures/NDF.png}
  \caption{Вплив функції розподілу нормалей}
  \label{fig:NDF}
\end{figure}

Таким чином, функція розподілу нормалей NDF визначає, наскільки вірогідною є наявність мікрофасета певної орієнтації в даній точці поверхні. Ця ймовірність 
безпосередньо впливає на вигляд відбитого світла, а отже -- і на візуальне сприйняття матеріалу.
\par 
  У моделі Cook-Torrance, NDF розраховується за формулою:
\begin{equation}
  D_{GGX}(\vec{h})=\frac{\alpha^2}{\pi(\langle\vec{n},\vec{h}\rangle^2(\alpha^2-1)+1)^2},
\end{equation}
де $\alpha = r_p^2$, а $r_p$ -- шорсткість матеріалу та приймає значення на відрізку $[0,1]$.

\subsection{Геометричний фактор}\\

\par
Останнім елементом мікрофасетної моделі є геометричний фактор, який враховує ефекти затінення (\textit{shadowing}) та перекриття (\textit{masking}) 
мікрофасетів. Ці ефекти виникають залежно від напрямку падіння світла та напрямку огляду спостерігача.

У роботі Кука і Торренса, передбачається, що мікрофасети мають форму літери V. 
На основі цієї моделі було виведено геометричний термін, який описує ймовірність того, що світло буде заблоковане іншими мікрофасетами 
або не зможе відбитися в напрямку спостерігача, обчислюється він наступним чином:
\begin{equation*}
  G_{Cook-Torrance}(\vec{l},\vec{v}) = min(1, \frac{2\langle\vec{n},\vec{h}\rangle\langle\vec{n},\vec{v}\rangle}
  {\langle\vec{v},\vec{h}\rangle},
  \frac{2\langle\vec{n},\vec{h}\rangle\langle\vec{n},\vec{l}\rangle}
  {\langle\vec{v},\vec{h}\rangle}).
\end{equation*}

Залежно від конфігурації, можливі три випадки: відсутність перешкод, \linebreak ефект перекриття при поглядових кутах під ковзаючим кутом, 
або затінення при падінні світла під аналогічним кутом (див. рис. \ref{fig:GeometryTerm}).Для нульового значення шорсткості (ідеально гладка поверхня) геометричний фактор дорівнює 1, 
тобто не впливає на результат. Зі збільшенням шорсткості значення геометричного фактору зменшується, 
що відповідає фізично очікуваному ефекту -- зростанню ймовірності перекриття і затінення мікрофасетів.

\begin{figure}[h]
  \centering
  \hspace*{-1.7cm} % або -0.1cm, -2mm тощо
  \includegraphics[scale=0.65]{Pictures/GeometryTerm.png}
  \caption{Вплив геометричного фактору}
  \label{fig:GeometryTerm}
\end{figure}



Іншу відому модель геометричного фактору запропонував Б.~Дж.~Сміт\linebreak (B.~J.~Smith) у 1967 році. 
У ній геометричний фактор обчислюється як добуток двох однакових функцій $G_1$, які розраховуються окремо для напрямку світла та напрямку огляду:

\[
G(l, v) = G_1(l) \cdot G_1(v).
\]

За функцію $G_1$ було обрано:
\begin{equation*}
  G(\vec{v}) = \frac{\sqrt{2}}{\sqrt{1 + \frac{\alpha(1-\langle\vec{n},\vec{v}\rangle)}{\langle\vec{n},\vec{v}\rangle}}}.
\end{equation*}
Таким чином, остаточне аналітичне представлення геометричного фактору, що враховує вплив мікроструктури поверхні на явища екранування та за\-ті\-нен\-ня, набуває наступного вигляду:

\begin{equation}
  \label{eq:GT}
G = \frac{2}{\sqrt{1 + \frac{\alpha(1-\langle\vec{n},\vec{v}\rangle)}{\langle\vec{n},\vec{v}\rangle}}
\sqrt{1 + \frac{\alpha(1-\langle\vec{n},\vec{l}\rangle)}{\langle\vec{n},\vec{l}\rangle}}}.
\end{equation}

Цей вираз є кульмінацією теоретичного узагальнення мікрофасетної геометрії, що водночас 
враховує складні взаємодії світлового потоку з текстурною нерівністю реальних поверхонь. Його 
застосування забезпечує фізично коректне моделювання поведінки світла на мікроскопічному рівні, що 
є критично важливим для створення фотореалістичних комп’ютерних зображень.

\par Такий підхід дозволяє точно моделювати як дифузні, так і дзеркальні відбиття, залежно від
оптичних характеристик матеріалу, зокрема показника заломлення, ступеня шорсткості та кута падіння 
світла. Завдяки цьому, система освітлення здатна відтворювати складну взаємодію світла з матеріалом -- 
від блискучих металів до матових діелектриків -- з урахуванням енергозбереження та фізичної правдоподібності.

\par У практичній реалізації мікрофасетної моделі критичним є вибір функції розподілу нормалей 
поверхні (NDF), таких як GGX або Beckmann, які визначають імовірнісний розподіл мікрофасеток по орієнтаціях. 
Це, в поєднанні з геометричними термінами самозатінення (G) та френелівськими коефіцієнтами (F), утворює 
повноцінну BRDF-модель (наприклад, Кука-Торренса), що дозволяє інтегрувати фізику світлових явищ 
у рендерінг-алгоритми сучасних графічних рушіїв.

\par Саме через ці характеристики мікрофасетна модель виступає основою у більшості сучасних 
технік фізично коректного рендерингу (PBR) і є фундаментом для багатьох подальших удосконалень, 
включаючи імплементацію попередньо обчислених таблиць (LUT), вибір за значимістю (англ. importance sampling), а також 
застосування моделей освітлення на основі оточення (IBL), що в сукупності дозволяє досягати високого 
візуального реалізму при збереженні продуктивності.
%         Розділ 2 
% *********************** Це є Розділ 3 ************************************

 \markright{\underline {\it Розділ 3. Чисельні експерименти }}

 \setcounter{chapter}{2}
 \chapter{Чисельні експерименти}
 У цьому розділі приведено чисельні експерименти
 
 \section{Середовище розробки та обладнання}
  \setcounter{equation}{0}
 \setcounter{theorem}{0}    

 Усі числові експерименти були виконані на ноутбуці Asus ROG Strix G15. Характеристики цього пристрою наведені нижче:
 
 \begin{itemize}
     \item \emph{GPU:} LAPTOP NVIDIA RTX 3060 з 6GB відеопам'яті
     \item \emph{CPU:} AMD Ryzen 7 6800H, 8 ядер, 16 потоків, з тактовою частотою до 4.7GHz
     \item \emph{Оперативна пам'ять:} 16GB DDR5
     \item \emph{SSD:} Kingston KC3000 з швидкістю читання/запису до 7GB/сек
 \end{itemize}
 
 \subsection{Середовище розробки}
 
 Для розробки та проведення числових експериментів використовувалися наступні середовища та інструменти:
 
 \begin{itemize}
     \item \emph{Visual Studio Code}\\
     Visual Studio Code (VS Code) — це потужний редактор коду, який підтримує безліч мов програмування і має великий набір плагінів для розширення функціональності. Він був використаний для написання та налагодження коду.
 
     \item \emph{Ще хтось}\\
 \end{itemize}
 
 \subsection{Використані бібліотеки}
 
 Для реалізації та тестування були використані наступні бібліотеки:
 
 \begin{itemize}
    \item \emph{Круті бібліотеки}\\
 \end{itemize}

\section{Результат чисельних експериментів}
  \setcounter{equation}{0}
 \setcounter{theorem}{0}%         Розділ 3 
%\include{D-ch4}%           Розділ 4 

%\include{xampl-ch4}%         Розділ 5 і т. д. ще скільки потрібно розділів



\markright{\underline{\it Висновки}}

\chapter*{Висновки}
\addcontentsline{toc}{chapter}{Висновки}
Дуже крутий висновок%       Висновки

\include{references}%         Список використаних джерел

%\appendix
%\include{xampl-app1}%        Додаток 1
%\include{xampl-app2}%        Додаток 2 і т. д. ще скільки потрібно додатків

% makeindex general  - запустити з Total Commander шоб оновило індекс

%\printindex{general}{}
%\printindex{general}{ПРЕДМЕТНИЙ ПОКАЖЧИК}

%\printindex{general}
%\printindex

%\input{general.ind}

  \markright{\underline
 {\it Список використаних джерел}}
 \addcontentsline{toc}{chapter}{Список використаних джерел}

% \large
 \renewcommand{\bibname}{{Список використаних джерел}}
 \begin{thebibliography}{}
 \setcounter{theorem}{0}

 \bibitem{Ch0}
  Сеньо П. С. Теорія ймовірностей та математична статистика / П. С. Сеньо - Київ : Знання, 2007. - 556 c.

  \bibitem{Ch1}
  The Colorimetric Properties of the Spectrum. 
  Philosophical Transactions of the Royal Society of London. Series A, Containing Papers of a Mathematical or Physical Character,
  Vol. 230 (1932), pp. 149-187 (39 pages)

  \bibitem{Ch2}
    CIE 1931 color space [Електронний ресурс]. – Режим доступу: \url{https://en.wikipedia.org/wiki/CIE_1931_color_space}.

   
  \bibitem{Ch3}
  Iosifidis A. Deep Learning for Robot Perception /  A. Iosifidis, A. Tefas - U.S: Academic Press, 2022. - 634 c.
    
    \bibitem{Ch4}
    UNDERSTANDING GAMMA CORRECTION [Електронний ресурс]. – Режим доступу: \url{https://www.cambridgeincolour.com/tutorials/gamma-correction.htm}.

    \bibitem{Ch5}
    UNDERSTANDING GAMMA CORRECTION [Електронний ресурс]. – Режим доступу: \url{https://www.cambridgeincolour.com/tutorials/gamma-correction.htm}.

    \bibitem{Ch6}
    Wikipedia. Partially Observable Markov Decision Process [Електронний ресурс]. – Режим доступу: \url{https://en.wikipedia.org/wiki/Partially_observable_Markov_decision_process}.

    \bibitem{Ch7}
    Tutorialspoint. Python Deep Learning - Deep Neural Networks [Електронний ресурс]. – Режим доступу: \url{https://www.tutorialspoint.com/python_deep_learning/python_deep_learning_deep_neural_networks.htm}.

    \bibitem{Ch8}
    Deepchecks. Reinforcement Learning Applications: From Gaming to Real World [Електронний ресурс]. – Режим доступу: \url{https://deepchecks.com/reinforcement-learning-applications-from-gaming-to-real-world/}.

    \bibitem{Ch9}
    Pacific Northwest National Laboratory (PNNL). Explainer: Deep Reinforcement Learning [Електронний ресурс]. – Режим доступу: \url{https://www.pnnl.gov/explainer-articles/deep-reinforcement-learning}.

    \bibitem{Ch10}
    Pathmind. Deep Reinforcement Learning [Електронний ресурс]. – Режим доступу: \url{https://wiki.pathmind.com/deep-reinforcement-learning}.

    \bibitem{Ch11}
    Neptune AI. Markov Decision Process in Reinforcement Learning [Електронний ресурс]. – Режим доступу: \url{https://neptune.ai/blog/markov-decision-process-in-reinforcement-learning}.

    \bibitem{Ch12}
    Built In. Markov Decision Process [Електронний ресурс]. – Режим доступу: \url{https://builtin.com/machine-learning/markov-decision-process}.
    
    \bibitem{Ch13}
    Towards Data Science. Understanding Actor-Critic Methods [Електронний ресурс]. – Режим доступу: \url{https://towardsdatascience.com/understanding-actor-critic-methods-931b97b6df3f}.
    
    \bibitem{Ch14}
    Scholarpedia. Policy Gradient Methods [Електронний ресурс]. – Режим доступу: \url{http://www.scholarpedia.org/article/Policy_gradient_methods#:~:text=Policy%20gradient%20methods%20are%20a,cumulative%20reward)%20by%20gradient%20descent}.
    
    \bibitem{Ch15}
    LessWrong. Deep Q-Networks Explained [Електронний ресурс]. – Режим доступу: \url{https://www.lesswrong.com/posts/kyvCNgx9oAwJCuevo/deep-q-networks-explained}.
    
 \end {thebibliography}



\end{document}








 
 
 

