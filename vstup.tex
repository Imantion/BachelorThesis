 \markright{\underline{\it Вступ}}

% \renewcommand{\baselinestretch}{1.3}
 \normalsize
 \chapter*{Вступ}
 \addcontentsline{toc}{chapter}{Вступ}
 %\large
У сучасному світі комп’ютерна графіка відіграє надзвичайно важливу роль. Вона присутня практично в кожному аспекті нашого цифрового життя: від інтерфейсів 
користувача на мобільних пристроях і комп’ютерах — до високобюджетного кінематографа та відеоігор. У багатьох випадках графіка використовується для візуалізації
 інформації або об'єктів без потреби в достовірному фізичному відтворенні взаємодії світла з матеріалами, як-от у веб-додатках чи користувацьких інтерфейсах. 
 Проте у сферах, де реалізм є ключовим — таких як віртуальна реальність, ігри чи CGI у фільмах — фізично коректне моделювання освітлення стає необхідністю. Саме
  від нього залежить глибина занурення глядача у віртуальний світ.

\par Реалістичне зображення не може бути досягнуте без розуміння та імітації фізичних процесів взаємодії світла з поверхнями. 
Однак комп’ютерна графіка стикається з низкою фундаментальних труднощів: фізичні явища є неперервними за своєю природою, а комп’ютери оперують 
дискретними структурами даних. Наприклад, постають питання, як описати складну геометрію тіла людини, як ефективно її візуалізувати, або як забезпечити
 правдоподібний рух цих об'єктів у просторі.

\par У цій науковій роботі увага буде зосереджена на одному з найважливіших аспектів — моделюванні фізично обґрунтованого освітлення. 
Інші компоненти графічного рушія, такі як геометричне моделювання чи анімація, будуть розглядатися лише побіжно або ігноруватимуться, оскільки 
метою роботи є реалізація та дослідження саме цього напряму.
 
