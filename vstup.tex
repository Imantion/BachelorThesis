 \markright{\underline{\it Вступ}}

% \renewcommand{\baselinestretch}{1.3}
 \normalsize
 \chapter*{Вступ}
 \addcontentsline{toc}{chapter}{Вступ}
 %\large
 Глибоке підкріплювальне навчання (Deep Reinforcement Learning, DRL) є 
 однією з найбільш перспективних галузей сучасного штучного інтелекту, 
 яка знаходить широке застосування у різних сферах, включаючи відеоігри. 
 Відеоігри є складними середовищами, де алгоритми DRL можуть демонструвати
  свою здатність до навчання складних стратегій та адаптації до змінних
   умов. Незважаючи на значні досягнення у розвитку DRL, проблема 
   ефективного навчання агентів у складних ігрових середовищах залишається
    актуальною.
 Актуальність даної роботи полягає в дослідженні можливостей різних
  моделей DRL для навчання агентів у грі Space Invaders, яка є класичним
   прикладом складної гри з багатьма викликами для агентів. 
   Розуміння того, яка з моделей (DQN, PPO, A2C) є найбільш 
   ефективною для цієї гри, допоможе вдосконалити методи навчання 
   агентів у подібних середовищах та сприятиме подальшому розвитку технологій DRL.
 
 Метою роботи є порівняння трьох популярних моделей DRL — DQN, PPO та 
 A2C — для визначення їхньої ефективності у навчанні агентів для гри 
 Space Invaders. Це порівняння включає аналіз їхньої продуктивності, 
 стабільності навчання та здатності до адаптації, що дозволить зробити
  вис\-новки про їхню придатність для вирішення складних ігрових завдань.
 
