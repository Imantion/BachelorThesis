\normalsize
\chapter*{ВСТУП}

У сучасному світі комп’ютерна графіка відіграє надзвичайно важливу роль. Вона присутня практично в кожному аспекті нашого цифрового життя: від інтерфейсів користувача на мобільних пристроях і комп’ютерах -- до ви\-со\-ко\-бюджет\-но\-го кінематографа та відеоігор. У багатьох випадках графіка використовується для візуалізації інформації або об'єктів без потреби в достовірному фізичному відтворенні взаємодії світла з матеріалами, як-от у веб-додатках чи користувацьких інтерфейсах.
Проте у сферах, де реалізм є ключовим -- таких як віртуальна реальність, ігри чи CGI у фільмах -- фізично коректне мо\-де\-лю\-ван\-ня освітлення стає необхідністю. Саме від нього залежить глибина за\-ну\-рен\-ня глядача у віртуальний світ і загальна переконливість зображення.

\par Реалістичне зображення не може бути досягнуте без розуміння та імітації фізичних процесів взаємодії світла з поверхнями. Однак комп’ютерна графіка стикається з низкою фундаментальних труднощів: фізичні явища є неперервними за своєю природою, а комп’ютери оперують дискретними структурами даних. Постають складні завдання: як описати геометрію складних об'єктів, як точно розрахувати їхню взаємодію зі світлом, як передати дрібні нюанси матеріалів -- усе це вимагає точних і водночас ефективних методів.

\par З огляду на це, важливою стає розробка та впровадження таких алгорит\-мів освітлення, які базуються на фізичних законах і водночас придатні до виконання в режимі реального часу. Саме ця задача і лягла в основу даної кваліфікаційної роботи.

\par Основна увага в роботі зосереджена на моделюванні фізично обґрунтованого освітлення в реальному часі. В межах дослідження реалізовано ключові компоненти системи рендерингу, які забезпечують коректну передачу ос\-віт\-лен\-ня та матеріалів на основі сучасних підходів. Інші аспекти графічного ру\-шія -- геометричне моделювання, анімація чи фізика -- розглядаються лише по\-біж\-но або не розглядаються взагалі.

\par Таким чином, \textit{об’єктом дослідження} виступає графічний рушій реального часу, що забезпечує побудову зоб\-ра\-жен\-ня шляхом рендерингу.
\textit{Предметом дослідження} є методи та алгоритми фізично коректного освітлен\-ня в цьому рушії, зокрема моделі відбиття для непрозорих тіл, та розсіювання світла, методи глобального освітлення, тіньоутворення, а також прийоми постобробки.

\par \textit{Актуальність роботи} зумовлена високими вимогами до якості зоб\-ра\-жен\-ня у сучасних візуальних застосунках. Із розвитком апаратного забезпечення дедалі більша увага приділяється не лише продуктивності, а й візуальній достовірності. Реалізація фізично обґрунтованих підходів до освітлення у режимі реального часу відкриває нові можливості для створення візуально переконливих сцен, водночас зберігаючи інтерактивність -- ключову вимогу для ігор та віртуальних симуляцій.