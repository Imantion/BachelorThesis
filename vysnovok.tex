\markright{\underline{\it Висновки}}

\chapter*{Висновки}
\addcontentsline{toc}{chapter}{Висновки}
У цій курсовій роботі ми вивчили та порівняли декілька моделей глибокого підкріплювального навчання (DRL) на прикладі гри Space Invaders. Дослідження показало, що кожна з розглянутих моделей - DQN, PPO і A2C - має свої переваги та недоліки.

Модель DQN виявилася досить ефективною у початкових стадіях тренування, проте стійкість до подальшого вдосконалення може бути обмеженою.

Модель PPO проявила найекспресивніші результати, здатність до яких розкрилася вже після початкової фази навчання. Проте виникає питання щодо стійкості цього підходу та його можливості до подальшого вдосконалення.

Нарешті, модель A2C показала стабільний та послідовний зріст у винагороді протягом усього тренування. Це свідчить про її здатність до пос\-тупового вдосконалення стратегій та великий потенціал для подальшого розвитку.

Отже, в контексті гри Space Invaders, модель A2C виглядає найбільш перспективною. Проте для конкретної гри або завдання важливо розглянути всі можливі аспекти та особливості кожної моделі перед прийняттям рішення щодо її використання.