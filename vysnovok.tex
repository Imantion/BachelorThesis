\chapter*{ВИСНОВКИ}
\par У цій кваліфікаційній роботі було здійснено комплексне дослідження тео\-ре\-тич\-них засад та практичної реалізації фізично коректного рендерингу (Phy\-si\-cal\-ly Based Rendering, PBR) у контексті тривимірної графіки та реального часу. Ос\-нов\-ною метою роботи було детальне вивчення сучасних моделей ос\-віт\-лен\-ня та матеріалів, а також побудова власного рендерера на основі графічного API DirectX~11 з підтримкою PBR-підходу.

\par Теоретична частина роботи зосереджена на фундаментальних поняттях розподілу світлових енергій у сцені, які описуються формальним рівнянням рендерингу. Було розглянуто властивості поверхонь на мікрорівні, зокрема мікрофасеткову модель відбиття світла, яка лежить в основі BRDF-функцій (Bi\-di\-rec\-tio\-nal Reflectance Distribution Function). Значну увагу приділено моделі Кука–Тор\-рен\-са, що поєднує у собі геометричну функцію видимості, розподіл нормалей поверхні (наприклад, GGX), а також ефект Френеля.

\par Крім того, розглянуто поділ матеріалів на метали та діелектрики, їхні відмінності у взаємодії з падаючим випромінюванням, а також роль таких параметрів, як металічність (metalness) та шорсткість (roughness). Описано особ\-ли\-вос\-ті дифузної  та дзеркальної  компонент відбиття, що забезпечують реаліс\-тич\-ну поведінку поверхні під різними кутами огляду та освітлення. 

\par У практичній частині реалізовано повноцінну систему PBR-рендерингу за допомогою DirectX~11. Було створено систему ресурсів для зчитування карт шорсткості, нормалей, металічності та албедо, організовано ефективну архі\-тек\-ту\-ру шейдерів, включно з модулем попередньої фільтрації environment map, а також генерації BRDF LUT для збереження продуктивності. Рендеринг виконувався за допомогою відкладеного затінювання (deferred shading), що дозволило ефективно обчислювати вплив освітлення для великої кількості пікселів одночасно.

\par У розділі чисельних експериментів було проведено серію візуальних досліджень, що демонструють поведінку матеріалів під впливом зміни параметрів PBR-моделі. Зокрема:
\begin{itemize}
    \item порівняння металічної сфери з різною шорсткістю (від ідеально гладкої до майже матової), що продемонструвало ефекти розмитого та чіт\-ко\-го дзеркального відбиття;
    \item демонстрація діелектричної сфери, де присутня лише дифузна складова без видимого дзеркального компоненту, що підтвердило правильність обчислення моделей матеріалів;
    \item пошарове відображення 3D-моделі самурая з різними компонентами освітлення (повне, лише дифузне, лише дзеркальне), що дозволило візуально оцінити внесок кожної складової BRDF-функції.
\end{itemize}

\par Усі отримані результати підтверджують правильність та ефективність реа\-лізованого підходу. Візуалізація об’єктів відповідає фізичним очікуванням й узгоджується з теоретичними моделями, що були розглянуті у попередніх розділах. Побудований рендерер продемонстрував здатність забезпечувати високу якість зображення при збереженні продуктивності в реальному часі.

\par У перспективі, розроблену систему можна розширити шляхом:
\begin{itemize}
    \item додавання глобального освітлення (Global Illumination) за допомогою технік, таких як Voxel Cone Tracing або Screen Space GI;
    \item інтеграції трасування променів у реальному часі (Ray Tracing), зокрема для обробки тіней, відбиттів та прозорості;
    \item моделювання підповерхневого розсіювання (Subsurface Scattering) для матеріалів на зразок шкіри або воску.
\end{itemize}
