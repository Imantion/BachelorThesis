% *********************** Це є Розділ 3 ************************************

 \markright{\underline {\it Розділ 3. Чисельні експерименти }}

 \setcounter{chapter}{2}
 \chapter{Чисельні експерименти}
 У цьому розділі приведено чисельні експерименти
 
 \section{Середовище розробки та обладнання}
  \setcounter{equation}{0}
 \setcounter{theorem}{0}    

 Усі числові експерименти були виконані на ноутбуці Asus ROG Strix G15. Характеристики цього пристрою наведені нижче:
 
 \begin{itemize}
     \item \emph{GPU:} LAPTOP NVIDIA RTX 3060 з 6GB відеопам'яті
     \item \emph{CPU:} AMD Ryzen 7 6800H, 8 ядер, 16 потоків, з тактовою частотою до 4.7GHz
     \item \emph{Оперативна пам'ять:} 16GB DDR5
     \item \emph{SSD:} Kingston KC3000 з швидкістю читання/запису до 7GB/сек
 \end{itemize}
 
 \subsection{Середовище розробки}
 
 Для розробки та проведення числових експериментів використовувалися наступні середовища та інструменти:
 
 \begin{itemize}
     \item \emph{Visual Studio Code}\\
     Visual Studio Code (VS Code) — це потужний редактор коду, який підтримує безліч мов програмування і має великий набір плагінів для розширення функціональності. Він був використаний для написання та налагодження коду.
 
     \item \emph{Ще хтось}\\
 \end{itemize}
 
 \subsection{Використані бібліотеки}
 
 Для реалізації та тестування були використані наступні бібліотеки:
 
 \begin{itemize}
    \item \emph{Круті бібліотеки}\\
 \end{itemize}

\section{Результат чисельних експериментів}
  \setcounter{equation}{0}
 \setcounter{theorem}{0}